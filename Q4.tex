\documentclass[11pt]{article}
\usepackage{amsmath}
\usepackage{amssymb}
\usepackage{mathrsfs}
\pagestyle{plain}
\usepackage{fullpage}
\usepackage{comment}
\includecomment{question}
\includecomment{solution}
\newcommand{\Implies}{\mbox{ IMPLIES }}
\newcommand{\Or}{\mbox{ OR }}
\newcommand{\AND}{\mbox{ AND }}
\newcommand{\Not}{\mbox{NOT}}
\newcommand{\Iff}{\mbox{ IFF }}
\newcommand{\True}{\mbox{T}}
\newcommand{\False}{\mbox{F}}
\def\reals{{\mathbb R}}
\def\ints{{\mathbb Z}}
\def\nats{{\mathbb N}}

% Environment for writing formal proofs======================================
%
% \nl ends line and makes next line a Numbered Line
% \ul ends line and makes next line an Unnumbered Line 
% \n increases the level of indentation ("next")
% \p decreases the level of indentation ("previous")
%\firstline  to number the first line of a proof
%
% Example:
% 
% 1     let $x \in \nats$ be arbitrary
% 2         let $y = x+1 \in \nats$
% 3         $y > x$; property of $\nats$, 2
% 4      $\exists y \in \nats. (y > x)$; construction 2, 3
% 5  $\forall x \in \nats. \exists y \in \nats. (y > x)$; generalization 1,4
%

% this can be typed as follows:
%
% \begin{formal}
% \firstline          <--- to number the first line of a proof
% \n \label{gen-start} let $x \in \nats$ be arbitrary \nl  
% \n let $y = x+1 \in \nats$ \label{defy} \nl
% $y > x$; property of $\nats$, \lref{defy} \label{gt} \nl
% \p $\exists y \in \nats. (y > x)$; \label{inside} construction \lref{defy},\lref{gt} \nl
% \p $\forall x \in \nats. \exists y \in \nats. (y > x)$; generalization \lref{gen-start}, \lref{inside}
% \end{formal}
% 
% You can use \labels anywhere in the code and \lref to refer to the line
% number.  (This is done using smartref package.)
%
%to reset the line numbering counter:
%\setcounter{linenum}{0} 

\usepackage{smartref} % for referencing the line numbers
\newcounter{linenum}
\addtoreflist{linenum}
\def\codeTabSpace{\hspace*{4mm}}
\newenvironment{formal}%
{\begin{tabbing}%
\codeTabSpace \= \hspace*{20mm} \= \hspace*{20mm} \= \hspace*{20mm} \= \kill%
}%
{\end{tabbing}%
}
\newcounter{ind}
\newcommand{\n}{\addtocounter{ind}{5}\hspace*{5mm}}
\newcommand{\p}{\addtocounter{ind}{-5}\hspace*{-5mm}}
\newcommand{\nl}{\\\stepcounter{linenum}{\scriptsize \arabic{linenum}}\>\hspace*{\value{ind}mm}}
\newcommand{\ul}{\\\>\hspace*{\value{ind}mm}}
\newcommand{\bl}{\\[-1.5mm]\>\hspace*{\value{ind}mm}}
\newcommand{\firstline}{\stepcounter{linenum}{\scriptsize \arabic{linenum}}\>}
\newcommand{\lref}[1]{\linenumref{#1}} % use this to refer to a line number
% End of stuff for entering formal=====================================

\begin{document}
\begin{center}
{\bf \Large \bf CSC240 Winter 2021 Midterm Assessment Question 4}\\
YOUR NAME and STUDENT NUMBER
\end{center}

\medskip

\begin{enumerate}
\setcounter{enumi}{3}

\item
\begin{question}
(15 marks)
Let $\mathscr{F}$ denote the set of all functions from $\nats$ to $\reals^+ \cup \{0\}$.\\
Recall that $\Omega(f') = \{g' \in \mathscr{F} \ |\ \exists c \in \reals^+.\exists b \in \nats. \forall n \in \nats.
[(n \geq b) \Implies (g'(n) \geq c \cdot  f'(n))]\}$.\\
For any functions $f \in \mathscr{F}$ and $g \in \mathscr{F}$,\\
%For $f,g \in \mathscr{F}$,
let $f+g$ denote the function $a \in \mathscr{F}$
where $a(n) = f(n)+g(n)$ for all $n \in \nats$ and\\
let $\max\{f,g\} \in \mathscr{F}$ denote the function $m \in \mathscr{F}$
where $m(n) = \max\{f(n),g(n)\}$ for all $n \in \nats$.

Formally prove $\forall f \in \mathscr{F}. \forall g \in \mathscr{F}.[ \max\{f,g\} \in \Omega(f+g)]$.

Number every line of your proof.
Explicitly state when a proof technique is being applied and say which earlier lines it refers to.\\
Use proper indentation. However,
to avoid excessive indentation, do not indent when making definitions.
\end{question}


\begin{solution}
{\bf Solution}:

\end{solution}
\end{enumerate}
\end{document}